% *======================================================================*
%  Cactus Thorn template for ThornGuide documentation
%  Author: Ian Kelley
%  Date: Sun Jun 02, 2002
%  $Header$
%
%  Thorn documentation in the latex file doc/documentation.tex
%  will be included in ThornGuides built with the Cactus make system.
%  The scripts employed by the make system automatically include
%  pages about variables, parameters and scheduling parsed from the
%  relevant thorn CCL files.
%
%  This template contains guidelines which help to assure that your
%  documentation will be correctly added to ThornGuides. More
%  information is available in the Cactus UsersGuide.
%
%  Guidelines:
%   - Do not change anything before the line
%       % START CACTUS THORNGUIDE",
%     except for filling in the title, author, date, etc. fields.
%        - Each of these fields should only be on ONE line.
%        - Author names should be separated with a \\ or a comma.
%   - You can define your own macros, but they must appear after
%     the START CACTUS THORNGUIDE line, and must not redefine standard
%     latex commands.
%   - To avoid name clashes with other thorns, 'labels', 'citations',
%     'references', and 'image' names should conform to the following
%     convention:
%       ARRANGEMENT_THORN_LABEL
%     For example, an image wave.eps in the arrangement CactusWave and
%     thorn WaveToyC should be renamed to CactusWave_WaveToyC_wave.eps
%   - Graphics should only be included using the graphicx package.
%     More specifically, with the "\includegraphics" command.  Do
%     not specify any graphic file extensions in your .tex file. This
%     will allow us to create a PDF version of the ThornGuide
%     via pdflatex.
%   - References should be included with the latex "\bibitem" command.
%   - Use \begin{abstract}...\end{abstract} instead of \abstract{...}
%   - Do not use \appendix, instead include any appendices you need as
%     standard sections.
%   - For the benefit of our Perl scripts, and for future extensions,
%     please use simple latex.
%
% *======================================================================*
%
% Example of including a graphic image:
%    \begin{figure}[ht]
% 	\begin{center}
%    	   \includegraphics[width=6cm]{MyArrangement_MyThorn_MyFigure}
% 	\end{center}
% 	\caption{Illustration of this and that}
% 	\label{MyArrangement_MyThorn_MyLabel}
%    \end{figure}
%
% Example of using a label:
%   \label{MyArrangement_MyThorn_MyLabel}
%
% Example of a citation:
%    \cite{MyArrangement_MyThorn_Author99}
%
% Example of including a reference
%   \bibitem{MyArrangement_MyThorn_Author99}
%   {J. Author, {\em The Title of the Book, Journal, or periodical}, 1 (1999),
%   1--16. {\tt http://www.nowhere.com/}}
%
% *======================================================================*

% If you are using CVS use this line to give version information
% $Header$

\documentclass{article}

% Use the Cactus ThornGuide style file
% (Automatically used from Cactus distribution, if you have a
%  thorn without the Cactus Flesh download this from the Cactus
%  homepage at www.cactuscode.org)
\usepackage{../../../../doc/latex/cactus}
\usepackage{amsmath}

\begin{document}

% The author of the documentation
\author{Miguel Zilh\~ao \textless mzilhao@ua.pt\textgreater}

% The title of the document (not necessarily the name of the Thorn)
\title{ScalarBase}

% the date your document was last changed, if your document is in CVS,
% please use:
%    \date{$ $Date$ $}
% when using git instead record the commit ID:
%    \date{\gitrevision{<path-to-your-.git-directory>}}
\date{July 1 2023}

\maketitle

% Do not delete next line
% START CACTUS THORNGUIDE

% Add all definitions used in this documentation here
% \def\mydef etc

% Add an abstract for this thorn's documentation
\begin{abstract}

ScalarBase adds an interface to include scalar fields within the Einstein
Toolkit framework. In the same spirit as ADMBase and HydroBase, this thorn
merely stores the variables used in typical scalar field evolutions as well as
commonly needed parameters. This thorn was primarily designed to work with the
ScalarEvolve thorn, but in principle any other scalar field evolution code
could be used.

\end{abstract}

\section{ScalarBase}

This thorn defines a set of variables and parameters commonly needed to evolve
(complex) scalar fields, potentially coupled to the Einstein field equations.
Currently it defines the gridfunctions \texttt{phi1} and \texttt{phi2}, as well
as their ``conjugated momenta'' \texttt{Kphi1} and \texttt{Kphi2}. It also
declares the parameter \texttt{mu}, the scalar field mass.

This thorn is designed to interface with the evolution thorn ScalarEvolve, but any other scalar evolution code should also work. Within ScalarEvolve, the variables \texttt{phi1} and \texttt{phi2} represent the real and imaginary parts of some complex scalar field $\Phi$, satisfying the Einstein-Klein-Gordon system
\begin{align}
   G_{\mu \nu} & = 8 \pi G T_{\mu \nu} \\
 \square \Phi & = \mu^2 \Phi
\end{align}
where $\mu$ is the mass parameter (denoted by \texttt{mu} in \texttt{param.ccl}) and
\begin{equation}
  T_{\mu \nu} = \bar \Phi_{,\mu} \Phi_{,\nu} + \Phi_{,\mu} \bar \Phi_{,\nu}
                - g_{\mu \nu} [  \bar \Phi^{,\sigma} \Phi_{,\sigma}
                               + \mu^2 \bar \Phi \Phi ]
\end{equation}


\section{Obtaining This Thorn}

This thorn is included in the Einstein Toolkit and can also be obtained through the \texttt{Canuda} numerical relativity library~\cite{Canuda}.

\begin{thebibliography}{9}

\bibitem{Canuda}
H.~Witek, M.~Zilhao, G.~Bozzola, C.-H.~Cheng, A.~Dima, M.~Elley, G.~Ficarra, T.~Ikeda, R.~Luna, C.~Richards, N.~Sanchis-Gual, H.~Okada~da~Silva.
``Canuda: a public numerical relativity library to probe fundamental physics,''
Zenodo (2023)
doi: 10.5281/zenodo.3565474

\end{thebibliography}

% Do not delete next line
% END CACTUS THORNGUIDE

\end{document}
