% *======================================================================*
%  Cactus Thorn template for ThornGuide documentation
%  Author: Ian Kelley
%  Date: Sun Jun 02, 2002
%  $Header$
%
%  Thorn documentation in the latex file doc/documentation.tex
%  will be included in ThornGuides built with the Cactus make system.
%  The scripts employed by the make system automatically include
%  pages about variables, parameters and scheduling parsed from the
%  relevant thorn CCL files.
%
%  This template contains guidelines which help to assure that your
%  documentation will be correctly added to ThornGuides. More
%  information is available in the Cactus UsersGuide.
%
%  Guidelines:
%   - Do not change anything before the line
%       % START CACTUS THORNGUIDE",
%     except for filling in the title, author, date, etc. fields.
%        - Each of these fields should only be on ONE line.
%        - Author names should be separated with a \\ or a comma.
%   - You can define your own macros, but they must appear after
%     the START CACTUS THORNGUIDE line, and must not redefine standard
%     latex commands.
%   - To avoid name clashes with other thorns, 'labels', 'citations',
%     'references', and 'image' names should conform to the following
%     convention:
%       ARRANGEMENT_THORN_LABEL
%     For example, an image wave.eps in the arrangement CactusWave and
%     thorn WaveToyC should be renamed to CactusWave_WaveToyC_wave.eps
%   - Graphics should only be included using the graphicx package.
%     More specifically, with the "\includegraphics" command.  Do
%     not specify any graphic file extensions in your .tex file. This
%     will allow us to create a PDF version of the ThornGuide
%     via pdflatex.
%   - References should be included with the latex "\bibitem" command.
%   - Use \begin{abstract}...\end{abstract} instead of \abstract{...}
%   - Do not use \appendix, instead include any appendices you need as
%     standard sections.
%   - For the benefit of our Perl scripts, and for future extensions,
%     please use simple latex.
%
% *======================================================================*
%
% Example of including a graphic image:
%    \begin{figure}[ht]
% 	\begin{center}
%    	   \includegraphics[width=6cm]{MyArrangement_MyThorn_MyFigure}
% 	\end{center}
% 	\caption{Illustration of this and that}
% 	\label{MyArrangement_MyThorn_MyLabel}
%    \end{figure}
%
% Example of using a label:
%   \label{MyArrangement_MyThorn_MyLabel}
%
% Example of a citation:
%    \cite{MyArrangement_MyThorn_Author99}
%
% Example of including a reference
%   \bibitem{MyArrangement_MyThorn_Author99}
%   {J. Author, {\em The Title of the Book, Journal, or periodical}, 1 (1999),
%   1--16. {\tt http://www.nowhere.com/}}
%
% *======================================================================*

% If you are using CVS use this line to give version information
% $Header$

\documentclass{article}

% Use the Cactus ThornGuide style file
% (Automatically used from Cactus distribution, if you have a
%  thorn without the Cactus Flesh download this from the Cactus
%  homepage at www.cactuscode.org)
\usepackage{../../../../doc/latex/cactus}
\usepackage{amsmath}

\begin{document}

% The author of the documentation
\author{Miguel Zilh\~ao \textless mzilhao@ua.pt\textgreater}

% The title of the document (not necessarily the name of the Thorn)
\title{ScalarEvolve}

% the date your document was last changed, if your document is in CVS,
% please use:
%    \date{$ $Date$ $}
% when using git instead record the commit ID:
%    \date{\gitrevision{<path-to-your-.git-directory>}}
\date{July 1 2023}

\maketitle

% Do not delete next line
% START CACTUS THORNGUIDE

% Add all definitions used in this documentation here
% \def\mydef etc

% Add an abstract for this thorn's documentation
\begin{abstract}

ScalarEvolve solves the (Einstein-)Klein-Gordon system of equations with 4th
order accurate finite-difference stencils. It relies on the MoL thorn
for the time integration and can work with any spacetime evolution thorn
to couple with the Einstein equations.

\end{abstract}


\section{ScalarEvolve}

This thorn provides the tools to evolve a complex scalar field with
nonlinear potential in arbitrary background as first described in
\cite{Scalar_Cunha:2017wao}. Additionally, it is possible to add an
external periodic forcing to the right-hand side of the Klein-Gordon equation
for studies of parametric resonances.

ScalarEvolve interfaces with the ScalarBase thorn, which defines the evolution
variables \texttt{phi1}, \texttt{phi2}, \texttt{Kphi1} and \texttt{Kphi2}.
%
This thorn merely takes care of setting the scalar field stress-energy tensor
and evolving (with MoL) the scalar field itself. The evolution of the metric
sector needs to be done elsewhere.

\section{Physical System}

Conventions used here are as in~\cite{Scalar_Cunha:2017wao} and as follows:
\begin{align}
  G_{\mu \nu} & = 8 \pi G T_{\mu \nu} \\
  \square \Phi & = \mu^2 \Phi
\end{align}
where $\mu$ is the mass parameter defined in \texttt{ScalarBase} and
\begin{equation}
  T_{\mu \nu} = \bar \Phi_{,\mu} \Phi_{,\nu} + \Phi_{,\mu} \bar \Phi_{,\nu}
                - g_{\mu \nu} [  \bar \Phi^{,\sigma} \Phi_{,\sigma}
                               + \mu^2 \bar \Phi \Phi ]
\end{equation}
and the gridfunctions \texttt{phi1} and \texttt{phi2} used in the code (defined in ScalarBase) are the real and imaginary part of the
complex field $\Phi = \phi_1 + i \phi_2$, which for reasons of convenience are evolved as separate, independent, variables.

Let us briefly present the $3+1$ decomposed equations of motion in the Einstein-Klein-Gordon case. To complete the characterization of the full spacetime we define the extrinsic curvature
\begin{equation}
\label{eq:KijDef}
K_{ij}  =   - \frac{1}{2\alpha} \left( \partial_{t} - \mathcal{L}_{\beta} \right) \gamma_{ij} \ ,
\end{equation}
and analogously introduce the ``canonical momentum'' of the complex scalar field $\Phi$
\begin{equation}
\label{eq:Kphi}
K_{\Phi} = -\frac{1}{2\alpha}  \left( \partial_{t} - \mathcal{L}_{\beta} \right) \Phi \,,
\end{equation}
where $\mathcal{L}$ denotes the Lie derivative.
%
Our evolution system can then be written in the form
%
\begin{align}
  \partial_{t} \gamma_{ij} & = - 2 \alpha K_{ij} + \mathcal{L}_{\beta} \gamma_{ij} \,,
                       \label{eq:dtgamma} \\
%
  \partial_{t} K_{ij}      & =  - D_{i} \partial_{j} \alpha
                       + \alpha \left( R_{ij} - 2 K_{ik} K^{k}{}_{j} + K K_{ij} \right) \nonumber \\
                       & \quad + \mathcal{L}_{\beta} K_{ij} 
                       % \nonumber \\  & \quad
                       + 4\pi \alpha \left[ (S-\rho) \gamma_{ij} - 2 S_{ij} \right] \,,
                                         \label{eq:dtKij} \\
% 
  \partial_{t} \Phi & = - 2 \alpha K_\Phi + \mathcal{L}_{\beta} \Phi\
                \,, \label{eq:dtPhi} \\
%
  \partial_{t} K_\Phi &  = \alpha \left( K K_{\Phi} - \frac{1}{2} \gamma^{ij} D_i \partial_j \Phi
                  + \frac{1}{2} \mu^2 \Phi \right) \nonumber \\
                 & \quad - \frac{1}{2} \gamma^{ij} \partial_i \alpha \partial_j \Phi
                       + \mathcal{L}_{\beta} K_\Phi \,, \label{eq:dtKphi}
\end{align}
% 
where $D_i$ is the covariant derivative with respect to the $3$-metric.


For completeness, we present also the full evolution equations re-written in the BSSN scheme~\cite{Scalar_Shibata:1995we,Scalar_Baumgarte:1998te}.
The full system of evolution equations is
%
\begin{align}
    \partial_t \tilde{\gamma}_{ij} & = \beta^k \partial_k \tilde{\gamma}_{ij} +
    2\tilde{\gamma}_{k(i} \partial_{j)} \beta^k - \frac{2}{3}
    \tilde{\gamma}_{ij} \partial_k \beta^k -2\alpha \tilde{A}_{ij},  \\
  %
    \partial_t \chi & = \beta^k \partial_k \chi + \frac{2}{3} \chi (\alpha K
    - \partial_k \beta^k),  \\
  % 
    \partial_t \tilde{A}_{ij} & = \beta^k \partial_k \tilde{A}_{ij}
      + 2\tilde{A}_{k(i} \partial_{j)} \beta^k
      - \frac{2}{3} \tilde{A}_{ij} \partial_k \beta^k  \notag \\
      & \quad + \chi \left( \alpha R_{ij} - D_i \partial_j \alpha\right)^{\rm
      TF}
      + \alpha \left( K\,\tilde{A}_{ij}
      - 2 \tilde{A}_i{}^k \tilde{A}_{kj} \right) \notag \\
      & \quad - 8 \pi \alpha \left(
          \chi S_{ij} - \frac{S}{3} \tilde \gamma_{ij}
        \right), \\
  % 
    \partial_t K & = \beta^k \partial_k K - D^k \partial_k \alpha + \alpha \left(
                   \tilde{A}^{ij} \tilde{A}_{ij} + \frac{1}{3} K^2 \right) \notag \\
    & \quad  + 4 \pi \alpha (\rho + S), \\
  %
    \partial_t \tilde{\Gamma}^i & = \beta^k \partial_k \tilde{\Gamma}^i
       - \tilde{\Gamma}^k \partial_k \beta^i + \frac{2}{3}
    \tilde{\Gamma}^i \partial_k \beta^k + 2 \alpha \tilde{\Gamma}^i_{jk}
                                  \tilde{A}^{jk} \notag \\
                                  & \quad + \frac{1}{3} \tilde{\gamma}^{ij}\partial_j \partial_k
    \beta^k
    + \tilde{\gamma}^{jk} \partial_j \partial_k \beta^i \nonumber \\
    & \quad - \frac{4}{3} \alpha \tilde{\gamma}^{ij} \partial_j K -
    \tilde{A}^{ij} \left( 3 \alpha \chi^{-1} \partial_j \chi + 2\partial_j
      \alpha \right) % -\left( \sigma + \frac{2}{3}\right) \left(\tilde{\Gamma}^i
      % -\tilde{\gamma}^{jk}\tilde{\Gamma}^i_{jk} \right) \partial_k \beta^k 
      % \notag \\ & \quad
                    \notag \\
    & \quad - 16 \pi \alpha \chi^{-1} j^i \label{eq:tilde-Gamma-evol} \,, \\
%
  \partial_{t} \Phi & = - 2 \alpha K_\Phi + \mathcal{L}_{\beta} \Phi\
                \,, \label{eq:dtPhi-BSSN} \\
%
  \partial_{t} K_\Phi &  = \alpha \left( K K_{\Phi} - \frac{1}{2} \gamma^{ij} D_i \partial_j \Phi
                  + \frac{1}{2} \mu^2 \Phi \right) \nonumber \\
                 & \quad - \frac{1}{2} \gamma^{ij} \partial_i \alpha \partial_j \Phi
                       + \mathcal{L}_{\beta} K_\Phi \,, \label{eq:dtKphi-BSSN}
\end{align}
%
with the source terms given by
%
\begin{align*}
  \rho & \equiv T^{\mu \nu}n_{\mu}n_{\nu} \,,\\
  j_i  &\equiv -\gamma_{i\mu} T^{\mu \nu}n_{\nu} \,, \\
  S_{ij} &\equiv \gamma^{\mu}{}_i \gamma^{\nu}{}_j T_{\mu \nu} \,, \\
  S     & \equiv \gamma^{ij}S_{ij} \,.
\end{align*}

Note however, as mentioned above, that the evolution of the metric sector is \emph{not} done in this thorn, and thus needs to be done elsewhere (ie, LeanBSSNMoL or McLachlan).


\section{Obtaining This Thorn}

This thorn is included in the Einstein Toolkit and can also be obtained through
the \texttt{Canuda} numerical relativity library~\cite{Canuda}.


\begin{thebibliography}{9}

\bibitem{Scalar_Cunha:2017wao}
P.~V.~P.~Cunha, J.~A.~Font, C.~Herdeiro, E.~Radu, N.~Sanchis-Gual and M.~Zilh\~ao,
``Lensing and dynamics of ultracompact bosonic stars,''
Phys. Rev. D \textbf{96}, no.10, 104040 (2017)
doi:10.1103/PhysRevD.96.104040
[arXiv:1709.06118 [gr-qc]].

\bibitem{Scalar_Shibata:1995we}
M.~Shibata and T.~Nakamura,
``Evolution of three-dimensional gravitational waves: Harmonic slicing case,''
Phys. Rev. D \textbf{52}, 5428-5444 (1995)
doi:10.1103/PhysRevD.52.5428

\bibitem{Scalar_Baumgarte:1998te}
T.~W.~Baumgarte and S.~L.~Shapiro,
``On the numerical integration of Einstein's field equations,''
Phys. Rev. D \textbf{59}, 024007 (1998)
doi:10.1103/PhysRevD.59.024007
[arXiv:gr-qc/9810065 [gr-qc]].

\bibitem{Canuda}
H.~Witek, M.~Zilhao, G.~Bozzola, C.-H.~Cheng, A.~Dima, M.~Elley, G.~Ficarra, T.~Ikeda, R.~Luna, C.~Richards, N.~Sanchis-Gual, H.~Okada~da~Silva.
``Canuda: a public numerical relativity library to probe fundamental physics,''
Zenodo (2023)
doi: 10.5281/zenodo.3565474

\end{thebibliography}

% Do not delete next line
% END CACTUS THORNGUIDE

\end{document}
