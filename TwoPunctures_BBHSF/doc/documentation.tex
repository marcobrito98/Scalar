% *======================================================================*
%  Cactus Thorn template for ThornGuide documentation
%  Author: Ian Kelley
%  Date: Sun Jun 02, 2002
%
%  Thorn documentation in the latex file doc/documentation.tex
%  will be included in ThornGuides built with the Cactus make system.
%  The scripts employed by the make system automatically include
%  pages about variables, parameters and scheduling parsed from the
%  relevant thorn CCL files.
%
%  This template contains guidelines which help to assure that your
%  documentation will be correctly added to ThornGuides. More
%  information is available in the Cactus UsersGuide.
%
%  Guidelines:
%   - Do not change anything before the line
%       % START CACTUS THORNGUIDE",
%     except for filling in the title, author, date, etc. fields.
%        - Each of these fields should only be on ONE line.
%        - Author names should be separated with a \\ or a comma.
%   - You can define your own macros, but they must appear after
%     the START CACTUS THORNGUIDE line, and must not redefine standard
%     latex commands.
%   - To avoid name clashes with other thorns, 'labels', 'citations',
%     'references', and 'image' names should conform to the following
%     convention:
%       ARRANGEMENT_THORN_LABEL
%     For example, an image wave.eps in the arrangement CactusWave and
%     thorn WaveToyC should be renamed to CactusWave_WaveToyC_wave.eps
%   - Graphics should only be included using the graphicx package.
%     More specifically, with the "\includegraphics" command.  Do
%     not specify any graphic file extensions in your .tex file. This
%     will allow us to create a PDF version of the ThornGuide
%     via pdflatex.
%   - References should be included with the latex "\bibitem" command.
%   - Use \begin{abstract}...\end{abstract} instead of \abstract{...}
%   - Do not use \appendix, instead include any appendices you need as
%     standard sections.
%   - For the benefit of our Perl scripts, and for future extensions,
%     please use simple latex.
%
% *======================================================================*
%
% Example of including a graphic image:
%    \begin{figure}[ht]
% 	\begin{center}
%    	   \includegraphics[width=6cm]{MyArrangement_MyThorn_MyFigure}
% 	\end{center}
% 	\caption{Illustration of this and that}
% 	\label{MyArrangement_MyThorn_MyLabel}
%    \end{figure}
%
% Example of using a label:
%   \label{MyArrangement_MyThorn_MyLabel}
%
% Example of a citation:
%    \cite{MyArrangement_MyThorn_Author99}
%
% Example of including a reference
%   \bibitem{MyArrangement_MyThorn_Author99}
%   {J. Author, {\em The Title of the Book, Journal, or periodical}, 1 (1999),
%   1--16. {\tt http://www.nowhere.com/}}
%
% *======================================================================*

\documentclass{article}

% Use the Cactus ThornGuide style file
% (Automatically used from Cactus distribution, if you have a
%  thorn without the Cactus Flesh download this from the Cactus
%  homepage at www.cactuscode.org)
\usepackage{../../../../doc/latex/cactus}


\newcommand{\bracket}[1]{\left( #1 \right)}

\begin{document}

% The author of the documentation
\author{Giuseppe Ficarra, Cheng-Hsin Cheng, Helvi Witek}

% The title of the document (not necessarily the name of the Thorn)
\title{TwoPunctures\_BBHSF}

% the date your document was last changed:
\date{Jun 24 2022}
\date{\today}

\maketitle

% Do not delete next line
% START CACTUS THORNGUIDE

% Add all definitions used in this documentation here
%   \def\mydef etc

% Add an abstract for this thorn's documentation
% \begin{abstract}

% \end{abstract}

% The following sections are suggestive only.
% Remove them or add your own.

\begin{abstract}
We extend the \texttt{TwoPunctures} thorn to generate initial data for black hole
binaries coupled to a massive, minimally-coupled complex scalar field.
\end{abstract}

\section{Scalar field source terms}

We consider a complex scalar field $\Phi$ of mass $\mu_S$ minimally coupled to gravity.
The action is described by
\begin{equation}
    S = \int \mathrm{d}^4 x \sqrt{-g}
    \bracket{ \frac{^{(4)}R}{16\pi} - \frac{1}{2} (\nabla\Phi)^2
        -\frac{\mu_S^2}{2} \Phi^2
    },
\end{equation}
resulting in the equations of motion
\begin{align}
    & G_{ab} = 8\pi T_{ab},
    \\
    & (g^{ab} \nabla_a \nabla_b -\mu_S^2) \Phi = 0,
\end{align}
with the stress-energy tensor
\begin{equation}
    T_{ab} = \nabla_a\Phi \nabla_b \Phi
    - \frac{1}{2} g_{ab} (\nabla^c\Phi \nabla_c \Phi + \mu_S^2\Phi^2 ).
\end{equation}

Under the 3+1 decomposition,
the scalar field contributes through the energy density $\rho$,
energy-momentum flux $j_i$,
and the purely spatial stress tensor $S_{ij}$, given by
\begin{align}
    \rho = n^\mu n^\nu T_{\mu\nu}
    &= 2 \Pi^2 + \frac{\mu_S^2 \Phi^2}{2}
    + \frac{1}{2} D_k\Phi D^k \Phi,
    \\
    j_i = -\gamma_i^\mu n^\nu T_{\mu\nu}
    &= 2\Pi D_i\Phi,
    \\
    S_{ij} = \gamma_i ^\mu \gamma_j^\nu T_{\mu\nu}
    &= D_i\Phi D_j\Phi
    + \frac{1}{2} \gamma_{ij} (4\Pi^2 - \mu_S^2 \Phi^2 - D^k\Phi D_k\Phi),
\end{align}
where $D_i$ is the spatial covariant derivative with respect to the 3-metric $\gamma_{ij}$.
Here, we have also defined the scalar field momentum $\Pi$ using the Lie derivative along the normal vector $n^\mu$ as
\begin{equation}
    \Pi \equiv - \frac{1}{2} \mathcal{L}_n \Phi.
\end{equation}

\section{Constraint equations in the punctures approach}

Let us perform a conformal decomposition of the physical metric as $\gamma_{ij}=\psi^4\bar{\gamma}_{ij}$,
where $\psi$ is the conformal factor and $\bar{\gamma}_{ij}$ is the conformal metric.
Then, the Hamiltonian constraint equation is given by
\begin{equation}
    \bar{D}^2 \psi - \frac{1}{8} \psi \bar{R} + \psi^5 (K_{ij}K^{ij} - K^2 )
    + 2\pi\psi^5 \rho = 0,
\end{equation}
where $\bar{D}^2$ is the conformal Laplacian operator, $\bar{R}$ is the conformal Ricci scalar, and $K_{ij}$ is the extrinsic curvature.
Further decomposing $K_{ij}$ into its trace $K$ and trace-free parts $A_{ij}$ and performing a conformal decomposition as
\begin{equation}
    A_{ij} = \psi^{-2}\bar{A}_{ij},
    \hfill
    A^{ij} = \psi^{-10}\bar{A}^{ij},
\end{equation}
we arrive at
\begin{equation}
    \bar{D}^2 \psi
    - \frac{1}{8} \psi \bar{R}
    + \frac{1}{8} \psi^{-7} \bar{A}_{ij}\bar{A}^{ij}
    - \frac{1}{12} \psi^5 K^2
    + 2\pi\psi^5 \rho = 0.
\end{equation}



Now let us introduce a series of assumptions for the punctures approach.
Writing the conformal factor as
$ \psi = u + \psi_{\rm BL} $
with the Brill-Lindquist solution
\begin{equation}
    \psi_{\rm BL} \equiv \sum_{i=1}^N \frac{m_i}{2|r - r_i|},
\end{equation}
where $m_i$ and $r_i$ denote the bare mass and location of the $i$-th puncture,
the flat Laplacian reduces to $\bar{D}^2 u$ in the punctured domain $\mathbb{R}^3\setminus\{r_i\}$.
Further assuming conformal flatness ($\bar{\gamma}_{ij} = \eta_{ij}$) and the maximal slicing condition ($K=0$), the Hamiltonian constraint equation takes the form of an elliptic equation in the unknown $u$
\begin{equation}
    \bar{D}^2 u + \frac{1}{8} (u+\psi_{\rm BL} )^{-7} \bar{A}_{ij} \bar{A}^{ij}
    + 2\pi (u+\psi_{\rm BL})^5 \rho = 0.
\end{equation}
In the above equation, the first two terms constitute the vacuum equations being solved in the \texttt{TwoPunctures} thorn.

\subsection{Conformal rescaling of scalar field}
Under the above set of assumptions, the energy density of the scalar field becomes
\begin{align}
    \rho
    &= 2 \Pi^2 + \frac{\mu_S^2 \Phi^2}{2}
    + \gamma^{ij} \frac{1}{2} D_i\Phi D_j \Phi
    \nonumber
    \\
    &= 2 \Pi^2 + \frac{\mu_S^2 \Phi^2}{2}
    + \psi^{-4} \frac{1}{2} \partial_k\Phi \partial^k \Phi.
\end{align}
If we choose a scalar field configuration for which the momentum vanishes at the initial time, for instance a Gaussian profile with zero momentum
\begin{equation}
    \Phi(t=0) = A_{\rm SF} Z(\theta, \phi) \exp[-(r-r_0)^2/w^2],
    \quad
    \Pi(t=0) = 0,
\end{equation}
the Hamiltonian constraint equation is now
\begin{equation}
    \bar{D}^2 u + \frac{1}{8} (u+\psi_{\rm BL} )^{-7} \bar{A}_{ij} \bar{A}^{ij}
    + \pi (u+\psi_{\rm BL})^5 \mu_S^2 \Phi^2
    + \pi (u+\psi_{\rm BL}) \partial_k\Phi \partial^k\Phi
    = 0.
\end{equation}

It would be nice if the equations end here.
Unfortunately, for certain choices of the scalar field mass $\mu_S$ the linearized equation can be an ill-posed elliptic problem,
leading to an unstable solution.
This motivates us to conformally rescale the scalar field as
\begin{equation}
    \Phi = \psi^\delta \bar{\Phi},
    \quad
    \Phi^* = \psi^\delta \bar{\Phi}^*,
\end{equation}
where $\delta$ is a constant parameter
and $\Bar{\Phi}$ is the conformal scalar field.
The Hamiltonian equation becomes
\begin{align}
    \nabla\psi
    + \frac{1}{8} \psi^{-7} \bar{A}_{ij} \bar{A}^{ij}
    +\
    &\pi\ \psi^{2\delta + 5}\
    \mu_S^2 \bar{\Phi}^* \bar{\Phi}
    \nonumber
    \\
    +\
    &\pi\ \psi^{2\delta + 1}\
    (\partial_i\bar{\Phi}^*)
    (\partial^i\bar{\Phi})
    \nonumber
    \\
    +\ \delta
    &\pi\ \psi^{2\delta}\quad\
    (\partial_i\psi)
    ( \bar{\Phi}^* \partial^i\bar{\Phi}
    + \bar{\Phi} \partial^i\bar{\Phi}^* )
    \nonumber
    \\
    +\ \delta^2
    &\pi\ \psi^{2\delta-1}\ (\partial_i\psi) (\partial^i\psi)
    \bar{\Phi}^* \bar{\Phi}
    \quad\quad\quad
    = 0,
\end{align}
where 
\begin{align}
    \psi 
    &= \psi_{\rm BL} + u,
    \\
    \partial_i\psi 
    &= \partial_i\psi_{\rm BL} + \partial_i u.
\end{align}

Linearizing it by writing the correction function as $u = u_0 + \epsilon$ with $u_0$ a known solution,
we arrive at
\begin{align}
    \nabla\epsilon
    - \epsilon
    \Bigg[
        \frac{7}{8} \psi^{-8} \bar{A}_{ij} \bar{A}^{ij}
        -\
        (2\delta + 5)\ \pi\
        &\psi^{2\delta + 4}\
        \mu_S^2 \bar{\Phi}^* \bar{\Phi}
        \nonumber
        \\
        -\ (2\delta + 1)\ \pi\
        &\psi^{2\delta}\quad\ 
        (\partial_i\bar{\Phi}^*)
        (\partial^i\bar{\Phi})
        \nonumber
        \\
        -\quad (2\delta)\quad \delta \pi\
        &\psi^{2\delta-1}\ (\partial_i\psi)
        ( \bar{\Phi}^* \partial^i\bar{\Phi}
        + \bar{\Phi} \partial^i\bar{\Phi}^* )
        \nonumber
        \\
        -\ (2\delta-1)\ \delta^2 \pi\
        &\psi^{2\delta-2}\ (\partial_i\psi) (\partial^i\psi)
        \bar{\Phi}^* \bar{\Phi}
    \quad\quad\quad\quad\Big]
    \nonumber
    \\
    +\
    (\partial_i\epsilon)\
    \Big[\
        \delta \pi\ \psi^{2\delta}\
        ( \bar{\Phi}^* \partial^i\bar{\Phi}
        &+ \bar{\Phi} \partial^i\bar{\Phi}^* )
        +
        \delta^2 \pi\ \psi^{2\delta-1}\ (\partial^i\psi)
        \bar{\Phi}^* \bar{\Phi}
    \
    \Big]= 0,
\end{align}
with
\begin{align}
    \psi
    &= \psi_{\rm BL} + u_0,
    \\
    \partial_i \psi
    &= \partial_i \psi_{\rm BL}
    + \partial_i u_0.
\end{align}


\section{Scalar field configurations}


\begin{thebibliography}{9}

%\cite{Liu:2009al}
\bibitem{Liu:2009al}
  Y.~T.~Liu, Z.~B.~Etienne and S.~L.~Shapiro,
  ``Evolution of near-extremal-spin black holes using the moving puncture technique,''
  Phys.\ Rev.\ D {\bf 80} (2009) 121503
  doi:10.1103/PhysRevD.80.121503
  [arXiv:1001.4077 [gr-qc]].
  %%CITATION = doi:10.1103/PhysRevD.80.121503;%%
  %18 citations counted in INSPIRE as of 26 Oct 2018

%\cite{Okawa:2014nda}
\bibitem{Okawa:2014nda}
  H.~Okawa, H.~Witek and V.~Cardoso,
  ``Black holes and fundamental fields in Numerical Relativity: initial data construction and evolution of bound states,''
  Phys.\ Rev.\ D {\bf 89} (2014) no.10,  104032
  doi:10.1103/PhysRevD.89.104032
  [arXiv:1401.1548 [gr-qc]].
  %%CITATION = doi:10.1103/PhysRevD.89.104032;%%
  %58 citations counted in INSPIRE as of 26 Oct 2018

% \cite{Zilhao:2015tya}
\bibitem{Zilhao:2015tya}
  M.~Zilh\~ao, H.~Witek and V.~Cardoso,
  ``Nonlinear interactions between black holes and Proca fields,''
  Class.\ Quant.\ Grav.\  {\bf 32} (2015) 234003
  doi:10.1088/0264-9381/32/23/234003
  [arXiv:1505.00797 [gr-qc]].
  %%CITATION = doi:10.1088/0264-9381/32/23/234003;%%
  %25 citations counted in INSPIRE as of 26 Oct 2018

\bibitem{Witek:2018tba}
  M.~Zilh\~ao, H.~Witek. To appear.


\end{thebibliography}

% Do not delete next line
% END CACTUS THORNGUIDE

\end{document}
